O cronograma apresentado na Tabela~\ref{tab:cronograma} organiza as principais atividades do projeto ao longo do ano de 2026, desde o levantamento inicial de requisitos até a apresentação final do trabalho. A distribuição temporal foi estruturada considerando a aplicação das metodologias ágeis propostas, com ciclos de desenvolvimento intercalados com etapas de testes e validação. As atividades foram planejadas de forma a garantir tempo adequado para cada fase do projeto, permitindo ajustes e refinamentos conforme necessário durante o processo de construção do sistema \textit{web} para acompanhamento de animais perdidos.

	\begin{table}[!htbp]
		\footnotesize %Habilitar se a tabela estiver muito comprida.
		\centering
		\caption{Cronograma de Atividades}
		\setlength{\tabcolsep}{0pt}
		\begin{tabular}{lcccccccccccccccccccc}
			
			\hline
			\multicolumn{ 1}{c}{\textbf{\cellcolor{gray!50}Atividade}} \cellcolor{gray!50}& & \multicolumn{19}{c}{\textbf{\cellcolor{gray!50}Período}}  \\ \cline{ 3 - 21}
			\cellcolor{gray!50}& \hspace{0.3cm}\cellcolor{gray!50} &
			\multicolumn{9}{c}{\textbf{\cellcolor{gray!50}Ano I}} & \multicolumn{10}{c}{\textbf{\cellcolor{gray!50}Ano II}} \\
			\cellcolor{gray!50}& \hspace{0.3cm}\cellcolor{gray!50} &
			\textbf{\cellcolor{gray!50}Fev}\cellcolor{gray!50} & \hspace{0.3cm}\cellcolor{gray!50} & \textbf{\cellcolor{gray!50}Mar} & \hspace{0.3cm}\cellcolor{gray!50} & \textbf{\cellcolor{gray!50}Abr} & \hspace{0.3cm}\cellcolor{gray!50} & \textbf{\cellcolor{gray!50}Mai} & \hspace{0.3cm}\cellcolor{gray!50} & \textbf{\cellcolor{gray!50}Jun} & \hspace{0.3cm}\cellcolor{gray!50} & \textbf{\cellcolor{gray!50}Ago} & \hspace{0.3cm}\cellcolor{gray!50} & \textbf{\cellcolor{gray!50}Set} & \hspace{0.3cm}\cellcolor{gray!50} & \textbf{\cellcolor{gray!50}Out} & \hspace{0.3cm}\cellcolor{gray!50} & \textbf{\cellcolor{gray!50}Nov} & \hspace{0.3cm}\cellcolor{gray!50} & \textbf{\cellcolor{gray!50}Dez} \\ \hline
			
			Levantamento de Requisitos & \hspace{0.3cm} & $\bullet$ & \hspace{0.3cm} & $\bullet$ & \hspace{0.3cm} &  & \hspace{0.3cm} &   & \hspace{0.3cm} &  & \hspace{0.3cm} &  & \hspace{0.3cm} &  & \hspace{0.3cm} &  & \hspace{0.3cm} &  & \hspace{0.3cm} &  \\
			
			\rowcolor{black!10}[0pt][0pt] Modelagem e Arquitetura do Sistema & \hspace{0.3cm} &  & \hspace{0.3cm} &  & \hspace{0.3cm} & $\bullet$  & \hspace{0.3cm} &   & \hspace{0.3cm} &  & \hspace{0.3cm} &  & \hspace{0.3cm} &  & \hspace{0.3cm} &  & \hspace{0.3cm} &  & \hspace{0.3cm} &  \\ 
			
			Prototipação das Interfaces & \hspace{0.3cm}&   & \hspace{0.3cm} &  & \hspace{0.3cm} & $\bullet$  & \hspace{0.3cm} &   & \hspace{0.3cm} &  & \hspace{0.3cm} &  & \hspace{0.3cm} &  & \hspace{0.3cm} &  & \hspace{0.3cm} &  & \hspace{0.3cm} &  \\ 
			
			\rowcolor{black!10}[0pt][0pt] Desenvolvimento da Aplicação & \hspace{0.3cm} &  & \hspace{0.3cm} &  & \hspace{0.3cm} &  & \hspace{0.3cm} & $\bullet$ & \hspace{0.3cm} & $\bullet$  & \hspace{0.3cm} & $\bullet$  & \hspace{0.3cm} &  & \hspace{0.3cm} &  & \hspace{0.3cm} &  & \hspace{0.3cm} & \\ 
			
			Testes e Validação & \hspace{0.3cm} &  & \hspace{0.3cm} &  & \hspace{0.3cm} &  & \hspace{0.3cm} & $\bullet$ & \hspace{0.3cm} & $\bullet$ & \hspace{0.3cm} & $\bullet$ & \hspace{0.3cm} & $\bullet$   & \hspace{0.3cm} &  & \hspace{0.3cm} &  & \hspace{0.3cm} & \\
			
			\rowcolor{black!10}[0pt][0pt]Escrita e Documentação & \hspace{0.3cm} &  & \hspace{0.3cm} &  & \hspace{0.3cm} &  & \hspace{0.3cm} & & \hspace{0.3cm} &  & \hspace{0.3cm} &  & \hspace{0.3cm} & & \hspace{0.3cm} & $\bullet$  & \hspace{0.3cm} & $\bullet$ & \hspace{0.3cm} &   \\ 
			
			Entrega e Defesa & \hspace{0.3cm} &  & \hspace{0.3cm} &  & \hspace{0.3cm} &  & \hspace{0.3cm} &   & \hspace{0.3cm} &  & \hspace{0.3cm} &  & \hspace{0.3cm} &  & \hspace{0.3cm} &  & \hspace{0.3cm} &  & \hspace{0.3cm} & $\bullet$  \\
			
			
		\end{tabular} \\
		\fonte {Elaboração Própria (2025)}
		\label{tab:cronograma}
	\end{table}

