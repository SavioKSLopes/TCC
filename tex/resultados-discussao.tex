\section{Resultados Esperados}

Por meio deste trabalho, busca-se oferecer uma solução tecnológica que facilite o reencontro entre animais perdidos e seus tutores, alcançando tanto pessoas que perderam seus \textit{pets} quanto aquelas que encontram animais nas ruas e desejam ajudar. Espera-se que o sistema se torne uma ferramenta acessível e intuitiva, capaz de ser utilizada por diferentes perfis de usuários, independentemente de sua familiaridade com tecnologia, tornando-se um recurso efetivo para a comunidade em geral.

Almeja-se reduzir significativamente o tempo de busca por animais desaparecidos e, consequentemente, diminuir o sofrimento emocional vivenciado pelos tutores durante esse período. Pretende-se também contribuir para a redução do número de animais circulando desabrigados nas ruas, promovendo impactos positivos tanto no bem-estar animal quanto na saúde pública das comunidades urbanas.

Do ponto de vista técnico, espera-se validar a aplicação de metodologias ágeis e \textit{Design Thinking} no desenvolvimento de sistemas voltados para problemas sociais, demonstrando que soluções tecnológicas centradas no usuário podem ser desenvolvidas de forma organizada e iterativa, gerando resultados práticos e funcionais.

Por fim, almeja-se que este trabalho sirva como referência acadêmica e prática para iniciativas similares, demonstrando a viabilidade de sistemas \textit{web} dedicados ao reencontro de animais e inspirando o desenvolvimento de ferramentas que unam tecnologia e responsabilidade social em benefício da comunidade e dos animais.