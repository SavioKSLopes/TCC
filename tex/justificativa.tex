O desaparecimento de animais de estimação no Brasil é significativamente mais comum do que se imagina: segundo levantamento de dados realizado pelo Instituto de Medicina Veterinária do Coletivo cerca de um terço dos cães e gatos já se perderam, mas apenas uma porcentagem reduzida é reencontrada (IMVC, 2024), o que evidencia a fragilidade dos métodos de busca existentes. Ao mesmo tempo, estimativas mais abrangentes indicam que há mais de 30 milhões de cães e gatos vivendo nas ruas ou em abrigos – um reflexo tanto do abandono quanto de desaparecimentos não solucionados.

Esse panorama é agravado pela ausência de repertório tecnológico adequado. Ferramentas genéricas de redes sociais, por si só, não oferecem funcionalidades essenciais como cadastro preventivo e comunicação em tempo real com possíveis encontradores. Isso contribui para a baixa taxa de reencontros e prolonga a angústia dos tutores.

Diante desse cenário, o desenvolvimento de um sistema \textit{web} especializado para monitoramento e reencontro de \textit{pets} perdidos cumpre vários propósitos, tais como: tecnológico, pois disponibiliza uma aplicação focada, segura e eficaz para localização de \textit{pets}; social, reduzindo o sofrimento de tutores e promovendo a guarda responsável; e científico, gerando dados consistentes sobre frequência, padrão e causas de desaparecimento, essenciais para orientar políticas públicas e campanhas educativas futuras.