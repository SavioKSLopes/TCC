A relação entre seres humanos e animais de estimação no Brasil tem passado por uma profunda transformação, consolidando os animais de estimação como figuras centrais no núcleo familiar. O Brasil alcançou a marca de 167,6 milhões de animais de estimação em 2023 \cite{IPB2024}, evidenciando a relevância cultural e social desses animais na sociedade contemporânea. Pesquisas indicam que 61\% dos tutores consideram seu animal de estimação como um membro da família \cite{CRMVSP2024}, fenômeno conhecido como humanização dos \textit{pets}, que resulta em uma configuração familiar conhecida como família multiespécie.

Essa profunda vinculação afetiva entre tutores e animais intensifica o sofrimento emocional vivenciado quando um animal de estimação desaparece. O desaparecimento não representa apenas a perda de um bem material, mas sim o rompimento de um vínculo familiar, desencadeando sintomas de tristeza profunda, culpa, ansiedade e até mesmo depressão nos tutores. Essa angústia é intensificada pela falta de respostas, pois a incerteza de não saber o que aconteceu com o animal pode ser devastadora.

O desaparecimento de animais de estimação é um problema recorrente no Brasil. Estimativas indicam que mais de 25\% dos cães e gatos estão abandonados ou perdidos \cite{IstoePet2024}, configurando um problema que afeta milhares de famílias anualmente. Além disso, as informações disponíveis apontam que um em cada três animais de estimação está desabrigado no mundo \cite{Mars2022}, evidenciando a magnitude da questão. Este cenário é ainda agravado pela ausência de soluções tecnológicas adequadas centralizadas que facilitem o reencontro entre animais perdidos e seus tutores.

Tradicionalmente, a busca por animais perdidos tem sido realizada através de métodos descentralizados e de alcance limitado, como publicações em redes sociais genéricas, cartazes físicos em postes e divulgação boca a boca. Essas abordagens apresentam diversas limitações, incluindo alcance geográfico restrito, dificuldade de organização das informações e ausência de ferramentas de geolocalização que poderiam agilizar significativamente o processo de busca.

Experiências internacionais demonstram que plataformas digitais dedicadas ao reencontro de animais perdidos podem aumentar substancialmente as chances de sucesso. A plataforma PawBoost afirma ter ajudado a reunir mais de 2 milhões de \textit{pets} perdidos com suas famílias \cite{PawBoost2024}. Adicionalmente, a parceria entre plataformas de auxílio e serviços de rede social hiperlocal permite que milhões de vizinhos recebam alertas imediatos quando um animal de estimação se perde na comunidade \cite{PetcoLoveLost2023}. Esses exemplos evidenciam o impacto positivo que plataformas \textit{web} podem ter quando aplicam notificações, geolocalização e mobilizam comunidades \textit{online}. No Brasil, entretanto, ainda existe carência de soluções tecnológicas estruturadas e acessíveis que centralizem essas informações e ofereçam recursos avançados de busca e identificação.

Nesse contexto, o desenvolvimento de um sistema \textit{web} específico para acompanhamento de animais perdidos apresenta-se como uma solução viável e necessária para enfrentar esse problema social. A proposta deste trabalho consiste na criação de uma plataforma que integre tecnologias contemporâneas de desenvolvimento \textit{web}, recursos de geolocalização e geração de \textit{QR Code} para identificação, oferecendo aos tutores uma ferramenta centralizada, acessível e eficiente para cadastro, busca e localização de animais desaparecidos.

O sistema proposto será desenvolvido utilizando metodologias ágeis, especificamente \textit{Scrum} e \textit{Kanban} \cite{atlassian2015scrum, miro2024kanban}, organizando o trabalho em \textit{Sprints} iterativos que permitirão entregas incrementais e ajustes baseados em \textit{feedback} contínuo. Adicionalmente, será aplicado o \textit{Design Thinking} nas fases iniciais do projeto \cite{ixdf2025design} para garantir compreensão empática das necessidades dos usuários, assegurando que as funcionalidades desenvolvidas estejam verdadeiramente alinhadas aos desafios enfrentados por tutores no processo de busca por animais perdidos.

O sistema também será desenvolvido considerando princípios de usabilidade e acessibilidade. A usabilidade refere-se aos atributos de qualidade que avaliam a facilidade de uso de interfaces de usuário \cite{Nielsen2020}, enquanto a acessibilidade assegura que o sistema funcione para todas as pessoas, independentemente de suas capacidades \cite{W3C2019}. Ambos os princípios são essenciais para garantir que o sistema seja efetivo no atendimento aos usuários em situações emocionais de estresse.

Este trabalho justifica-se pela relevância social do problema abordado, pelo potencial de redução do sofrimento emocional dos tutores através de ferramentas tecnológicas eficazes, e pela contribuição para a diminuição do número de animais circulando desabrigados nas ruas, com consequentes benefícios para a saúde pública e bem-estar animal. Além disso, o projeto demonstra a aplicabilidade de metodologias ágeis no desenvolvimento de soluções voltadas para problemas comunitários, podendo servir como referência para iniciativas similares.

A estrutura deste trabalho está organizada da seguinte forma: esta introdução apresenta a contextualização do problema. Os Objetivos são apresentados no Capítulo 2. O Capítulo 3 expõe a Justificativa. A Revisão Bibliográfica encontra-se no Capítulo 4, abordando conceitos sobre animais de estimação no Brasil, o problema do desaparecimento, plataformas digitais existentes, metodologias ágeis, usabilidade e acessibilidade. O Capítulo 5 descreve detalhadamente a Metodologia utilizada, incluindo o delineamento da pesquisa, participantes, levantamento de requisitos, equipamentos, recursos, procedimentos de coleta e análise de dados. O Capítulo 6 apresenta os Resultados Esperados. Por fim, o Capítulo 7 contém o Cronograma de atividades. 