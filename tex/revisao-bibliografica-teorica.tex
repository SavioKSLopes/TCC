\section{Animais de Estimação no Brasil}

Animais de estimação, também chamados de animais de companhia, são aqueles mantidos por seres humanos em ambiente doméstico ou próximo dele, com fins principalmente afetivos, de convívio ou companhia, e não para produção de alimento ou trabalho. Em muitos estudos de psicologia e sociologia, esses animais ocupam espaço simbólico de membros da família, recebendo cuidados, atenção e afeto. Um trabalho sobre convivência humano-animal afirma que “o presente estudo [...] buscou compreender a influência da convivência com animais de estimação na vida das pessoas a partir da percepção de seus tutores” (Giumelli, 2016). 

Nesse contexto, a posse responsável é um conceito importante: implica cuidados sanitários, alimentação, atenção veterinária, espaço adequado e convívio social. A ampliação desse conceito nos últimos anos também advém da mudança nos valores culturais — muitas pessoas veem seus pets como “filhos de quatro patas”.

Quantidade de animais de estimação no Brasil

O Brasil figura entre os países com maior número de pets no mundo. Segundo dados da Abinpet e do Instituto Pet Brasil, o país possui cerca de 160,9 milhões de animais de estimação (cães, gatos, aves, répteis, peixes e pequenos mamíferos). 

Outros levantamentos e estimativas corroboram números próximos. Por exemplo:

\section{O Desaparecimento de Animais de Estimação no Brasil}

O desaparecimento de animais de estimação é um problema recorrente no Brasil e no mundo. Estudo conduzido pela Mars Petcare em 2022 apontou que “um em cada três animais de estimação está desabrigado no mundo” (Mars, 2022). Esse dado evidencia a magnitude da questão e reforça a necessidade de soluções que facilitem a comunicação entre tutores e comunidade. Além do sofrimento emocional e da atribulação enfrentada pelos tutores, a perda de animais pode gerar impactos sociais e de saúde pública. Animais soltos nas ruas podem se tornar vetores de zoonoses, contribuir para acidentes de trânsito e agravar problemas sanitários, sobretudo em áreas urbanas com grande densidade populacional.

No Brasil, a situação também é alarmante. Estimativas recentes mostram que “mais de 25\% dos cães e gatos estão abandonados ou perdidos” (Valor Econômico, 2024). Esses números indicam não apenas a dimensão do problema, mas também a necessidade de políticas públicas e soluções tecnológicas inovadoras. Socialmente, o abandono de \textit{pets} também reflete a fragilidade de valores relacionados à responsabilidade e ao cuidado. (UFPB, 2024). Esse cenário justifica a adoção de ferramentas digitais que otimizem a busca por animais desaparecidos e ampliem as possibilidades de reencontro entre tutores e seus animais, além de uma forma de informatização da sociedade sobre o problema que o abandono se torna neste contexto.

\section{Plataformas digitais e sistemas web}

Nos últimos anos, cresceram as iniciativas digitais para auxiliar no reencontro de animais, aproveitando o alcance das redes sociais e o avanço de tecnologias web. A plataforma PawBoost, por exemplo, afirma: “Nós ajudamos a reunir mais de 2 milhões de \textit{pets} perdidos com suas famílias” (PawBoost, 2024). Esse exemplo demonstra o impacto positivo que plataformas web podem ter quando aplicam notificações, geolocalização e mobilizam comunidades online. A eficácia dessas soluções está justamente em seu poder de rede, pois permitem que centenas de pessoas recebam informações em tempo real, aumentando a probabilidade de localização rápida.

Outra iniciativa relevante é a parceria entre a plataforma de auxílio Petco Love Lost e a Nextdoor (empresa que opera um serviço de rede social hiperlocal para bairros), que “permite que milhões de vizinhos recebam alertas imediatos quando um animal de estimação se perde na comunidade” (Petco Love Lost, 2023). Essa integração mostra como sistemas web podem usar dados de vizinhança para potencializar o alcance das buscas, criando um ecossistema colaborativo entre tutores, vizinhos e organizações de proteção animal. Como destaca Castells (2003), “a internet tornou-se a principal infraestrutura de comunicação em rede, transformando a forma como as sociedades funcionam”. No contexto do reencontro de animais, isso significa usar a conectividade digital para enfrentar um problema tradicional com novos recursos.

\section{Usabilidade e acessibilidade em sistemas web}
Um sistema web destinado ao reencontro de animais deve priorizar simplicidade e acessibilidade, dado que os tutores se encontram em situações emocionais de estresse. Nielsen (1994) estabelece que “a usabilidade refere-se a atributos de qualidade que avaliam a facilidade de uso de interfaces de usuário” (NIELSEN NORMAN GROUP, 2020). Isso significa que a interface deve ser intuitiva, de fácil aprendizado e fornecer feedback claro em todas as etapas, desde o cadastro do animal até a emissão de alertas de desaparecimento.

A acessibilidade também é obrigatória e se relaciona diretamente com o princípio da inclusão digital. A W3C afirma que “a Web é fundamentalmente projetada para funcionar para todas as pessoas, independentemente de seu hardware, software, idioma, cultura, localização ou capacidade” (W3C, 2019). Para tanto, recomenda-se que sistemas web adotem boas práticas como textos alternativos em imagens, contraste adequado entre texto e fundo, navegação por teclado e design responsivo. Segundo a norma ISO 9241-11:2018, “a usabilidade é a extensão na qual um sistema pode ser usado por usuários específicos para alcançar objetivos específicos com eficácia, eficiência e satisfação em um contexto de uso especificado” (ISO, 2018). Isso reforça a importância de realizar testes de usabilidade com o público-alvo, avaliando o desempenho real do sistema em situações práticas.

Outro ponto importante é o fator emocional. Estudos em Interação Humano-Computador destacam que sistemas projetados para contextos de estresse devem ser ainda mais claros, reduzindo frustração e aumentando a confiança do usuário. “O design emocional é essencial porque produtos que evocam confiança e empatia melhoram a experiência de uso”(Norman; Nielsen, 2013).

\section{Proteção de dados e legislação}
O tratamento de dados pessoais em sistemas web no Brasil é regido pela Lei Geral de Proteção de Dados Pessoais (LGPD). O Artigo 1º da Lei nº 13.709/2018 estabelece: “Esta Lei dispõe sobre o tratamento de dados pessoais, inclusive nos meios digitais, por pessoa natural ou por pessoa jurídica de direito público ou privado, com o objetivo de proteger os direitos fundamentais de liberdade e de privacidade” (Planalto, 2018).

Isso implica que o sistema web deve obter consentimento explícito dos usuários para coleta de dados, além de oferecer meios claros de exclusão e atualização das informações. O Artigo 18 complementa: “O titular dos dados pessoais tem direito a obter do controlador […] a correção de dados incompletos, inexatos ou desatualizados” (Planalto, 2018). Em outras palavras, qualquer sistema que lide com dados de tutores, como nome, telefone, e-mail e localização do animal, precisa garantir mecanismos de segurança e transparência.

A LGPD internaliza a orientação constitucional de que a proteção do consumidor e a dignidade da pessoa humana são erigidas como princípios da ordem econômica . As suas disposições preliminares enunciam que a disciplina da proteção de dados pessoais tem como objetivo proteger os direitos fundamentais e o livre desenvolvimento da personalidade (art. 1º), repetindo-os como um dos seus fundamentos ao lado do desenvolvimento econômico-tecnológico e da inovação (art. 2º)(BIONI; Bruno, 2019). No caso de um sistema para reencontro de animais, é essencial equilibrar o compartilhamento de informações (necessário para localizar o animal) com a proteção da privacidade dos usuários.

\section{Considerações finais}
A revisão de literatura evidencia que o uso de sistemas web dedicados ao reencontro de animais perdidos aumenta significativamente as chances de sucesso em comparação com métodos tradicionais. Exemplos internacionais como PawBoost e Petco Love Lost demonstram que o uso de geolocalização, alertas comunitários e redes de vizinhança são práticas eficazes. No Brasil, iniciativas locais podem se beneficiar dessa abordagem, desde que respeitem princípios de usabilidade, acessibilidade e a legislação vigente sobre dados pessoais.

Assim, a fundamentação teórica mostra que há espaço para a construção de uma solução inovadora: um sistema web que centralize informações sobre animais desaparecidos, integre funcionalidades de alerta, garanta acessibilidade universal e respeite integralmente a LGPD. Ao unir tecnologia e responsabilidade social, esse tipo de sistema tem potencial para reduzir o tempo de reencontro, diminuir impactos emocionais nos tutores e colaborar com a sociedade no enfrentamento do problema do desaparecimento de animais.