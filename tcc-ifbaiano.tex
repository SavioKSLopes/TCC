% ------------------------------------------------------------------------
% ------------------------------------------------------------------------
% ICMC: Modelo de Trabalho Acadêmico (tese de doutorado, dissertação de
% mestrado e trabalhos monográficos em geral) em conformidade com 
% ABNT NBR 14724:2011: Informação e documentação - Trabalhos acadêmicos -
% Apresentação
% ------------------------------------------------------------------------
% ------------------------------------------------------------------------

% Opções: 
%   Qualificação          = qualificacao 
%   Curso                 = doutorado/mestrado
%   Situação do trabalho  = pre-defesa/pos-defesa (exceto para qualificação)
%   Versão para impressão = impressao
\documentclass[qualificacao]{packages/icmc}

% ---------------------------------------------------------------------------
% Pacotes Opcionais
% ---------------------------------------------------------------------------
\usepackage{rotating}           % Usado para rotacionar o texto
\usepackage[all,knot,arc,import,poly]{xy}   % Pacote para desenhos gráficos
% Este pacote pode conflitar com outros pacotes gráficos como o ``pictex''
% Então é necessário usar apenas um dos pacotes conflitantes
\newcommand{\VerbL}{0.52\textwidth}
\newcommand{\LatL}{0.42\textwidth}

% ---------------------------------------------------------------------------







% ---
% Informações de dados para CAPA e FOLHA DE ROSTO
% ---
% Tanto na capa quanto nas folhas de rosto apenas a primeira letra da primeira palavra (ou nomes próprios) devem estar em letra maiúscula, todas as demais devem ser em letra minúscula.
\tituloPT{Sistema para acompanhamento de animais perdidos}
\tituloEN{System for Tracking Lost Animals}
\autor[Último, P. S.]{Sávio Kauan Silveira Lopes}
\genero{M} % Gênero do autor (M = Masculino / F = Feminino)
\orientador[Orientador]{Prof. Dr.}{George Gabriel Mendes Dourado}
%\coorientador{Prof. Dr.}{Fulano de Tal}
\curso{CCMC}
\data{29}{06}{2019} % Data do depósito
\idioma{PT} % Idioma principal do documento (PT = português / EN = inglês)
% ---



% ---
% RESUMOS
% ---

% Resumo em PORTUGUÊS
% conter no máximo 500 palavras
% conter no mínimo 1 e no máximo 5 palavras-chave
\textoresumo[brazil]{
     Este trabalho é um breve modelo  para a escrita de monografias de qualificação, dissertações e teses utilizando o ambiente \LaTeX, de acordo com as normas exigidas pelo Instituto de Ciências Matemáticas e de Computação (ICMC), da Universidade de São Paulo (USP). Para a confecção deste modelo foi utilizado a última versão (1.9.6) do pacote de classes \textit{abnTeX2} que segue as normas da Associação Brasileira de Normas Técnicas. A elaboração de uma monografia, dissertação ou tese pode ser feita sobrescrevendo o conteúdo deste modelo. 
}{Modelo, Monografia de qualificação, Dissertação, Tese, Latex}


% resumo em INGLÊS
% conter no máximo 500 palavras
% conter no mínimo 1 e no máximo 5 palavras-chave
\textoresumo[english]{
     This paper is a brief model for writing qualification monographs, dissertations and thesis using \LaTeX environment, in accordance with the standards required by the Institute of Mathematics and Computer Sciences (ICMC), University of São Paulo (USP). For making this model, the latest version (1.9.6) \textit{abnTeX2} classes package was used. This package follow the rules of the Brazilian Association of Technical Standards. A drafting a monograph, dissertation or thesis can be done by overwriting the contents of this model.
}{Template, Qualification monograph, Dissertation, Thesis, Latex}


% ----------------------------------------------------------
% ELEMENTOS PRÉ-TEXTUAIS
% ----------------------------------------------------------


% Inserir a ficha catalográfica
\incluifichacatalografica{tex/pre-textual/ficha-catalografica/ficha-catalografica}

% DEDICATÓRIA / AGRADECIMENTO / EPÍGRAFE
\textodedicatoria*{tex/pre-textual/dedicatoria/dedicatoria}
\textoagradecimentos*{tex/pre-textual/agradecimentos/agradecimentos}
\textoepigrafe*{tex/pre-textual/epigrafe/epigrafe}

% Inclui a lista de figuras
\incluilistadefiguras

% Inclui a lista de tabelas
\incluilistadetabelas

% Inclui a lista de quadros
\incluilistadequadros

% Inclui a lista de algoritmos
\incluilistadealgoritmos

% Inclui a lista de códigos
\incluilistadecodigos

% Inclui a lista de siglas e abreviaturas
\incluilistadesiglas

% Inclui a lista de símbolos
\incluilistadesimbolos

% ----
% Início do documento
% ----
\begin{document}
% ----------------------------------------------------------
% ELEMENTOS TEXTUAIS
% ----------------------------------------------------------
\textual

\chapter{Introdução}
\label{chapter:introducao}
\sigla{TICs}{Tecnologias de Informação e Comunicação}

\sigla{IFBAIANO}{Instituto Federal de Educação, Ciência e Tecnologia Baiano}



Testando \citeonline{barwaldt:2008}.  


Testando \cite{barwaldt:2008}. 

 \begin{citacao}
 	As citações diretas, no texto, com mais de três linhas [...] deve-se
 	observar apenas o recuo \cite[p.78]{naidson:2017}.
 \end{citacao}

Testando dentro do texto \citeonline{EspiritoSanto1987}

Testando no final do texto \cite{EspiritoSanto1987}


Testando \cite{DeRose2015}

Testando \citeonline[p.26]{DeRose2015}


Testando dois autores \cite{Novak1967}


Testando no final do texto \cite{Smith2014}


Testando dois autores \citeonline{Dias2014}

Testando três autores.... citação direta.


\begin{citacao}
	As citações diretas, no texto, com mais de três linhas [...] deve-se
	observar apenas o recuo \cite[p.78]{Giannini2000}.
\end{citacao} 

Testando três autores \citeonline{Giannini2000}


Testando quatro autores \citeonline{Pasquarelli1987}.

Testando quatro autores \cite{an:2013}


Testando quatro autores \cite{huegel:2015}

Responsabilidade pelo conjunto da obra. \cite{delvecchio1995}


\cite{apae:2006}

\cite{carniel:2015}

\cite{brasil:2008}

\citeonline{barbosa:2016}



\chapter{Objetivos} 
\label{chapter:objetivos}
\section{Objetivo Geral}
Desenvolver um sistema \textit{mobile} para auxiliar na busca por animais perdidos, permitindo o cadastro prévio para monitoramento e a divulgação eficiente dos desaparecimentos, facilitando a comunicação entre donos e a população e ampliando as chances de reencontro do \textit{pet} com rapidez e segurança.



\section{Objetivos Específicos}
\begin{itemize}
	\item Realizar um levantamento de requisitos do sistema a ser desenvolvido, baseando-se na adaptação e na melhoria das estratégias não computacionais adotadas na procura por animais perdidos e nas principais demandas atuais do publico-alvo do sistema;
	\item Modelar o sistema de acompanhamento de animais perdidos, definindo suas entidades, relacionamentos e fluxos de informação para garantir um funcionamento consistente.
	\item Implementar o \textit{front-end}, \textit{back-end} e banco de dados do sistema, além de realizar integrações necessárias com \textit{API}s externas;
	\item Executar testes de unidade, integração e sistema para assegurar a consistência, o funcionamento correto e o desempenho da aplicação;
	\item Validar o sistema com usuários-alvo por meio de testes de validação, comparando a solução proposta com métodos tradicionais e coletando \textit{feedback} para melhorias.
\end{itemize}


\chapter{Definição do Problema}
\label{chapter:definicao-do-problema}
\input{tex/definicao-do-problema}

\chapter{Hipótese}
\label{chapter:hipotese}
\input{tex/hipotese}

\chapter{Justificativa}
\label{chapter:justificativa}
O desaparecimento de animais de estimação no Brasil é significativamente mais comum do que se imagina: segundo levantamento de dados realizado pelo Instituto de Medicina Veterinária do Coletivo cerca de um terço dos cães e gatos já se perderam, mas apenas uma porcentagem reduzida é reencontrada (IMVC, 2024), o que evidencia a fragilidade dos métodos de busca existentes. Ao mesmo tempo, estimativas mais abrangentes indicam que há mais de 30 milhões de cães e gatos vivendo nas ruas ou em abrigos – um reflexo tanto do abandono quanto de desaparecimentos não solucionados.

Esse panorama é agravado pela ausência de repertório tecnológico adequado. Ferramentas genéricas de redes sociais, por si só, não oferecem funcionalidades essenciais como cadastro preventivo e comunicação em tempo real com possíveis encontradores. Isso contribui para a baixa taxa de reencontros e prolonga a angústia dos tutores.

Diante desse cenário, o desenvolvimento de um sistema \textit{web} especializado para monitoramento e reencontro de \textit{pets} perdidos cumpre vários propósitos, tais como: tecnológico, pois disponibiliza uma aplicação focada, segura e eficaz para localização de \textit{pets}; social, reduzindo o sofrimento de tutores e promovendo a guarda responsável; e científico, gerando dados consistentes sobre frequência, padrão e causas de desaparecimento, essenciais para orientar políticas públicas e campanhas educativas futuras.

\chapter{Revisão Bibliográfica/Teórica}
\label{chapter:revisao-bibliografica-teorica}
\section{Animais de Estimação no Brasil}

Animais de estimação, também chamados de animais de companhia, são aqueles mantidos por seres humanos em ambiente doméstico ou próximo dele, com fins principalmente afetivos, de convívio ou companhia, e não para produção de alimento ou trabalho. Em muitos estudos de psicologia e sociologia, esses animais ocupam espaço simbólico de membros da família, recebendo cuidados, atenção e afeto. Um trabalho sobre convivência humano-animal afirma que “o presente estudo [...] buscou compreender a influência da convivência com animais de estimação na vida das pessoas a partir da percepção de seus tutores” (Giumelli, 2016). 

Nesse contexto, a posse responsável é um conceito importante: implica cuidados sanitários, alimentação, atenção veterinária, espaço adequado e convívio social. A ampliação desse conceito nos últimos anos também advém da mudança nos valores culturais — muitas pessoas veem seus pets como “filhos de quatro patas”.

Quantidade de animais de estimação no Brasil

O Brasil figura entre os países com maior número de pets no mundo. Segundo dados da Abinpet e do Instituto Pet Brasil, o país possui cerca de 160,9 milhões de animais de estimação (cães, gatos, aves, répteis, peixes e pequenos mamíferos). 

Outros levantamentos e estimativas corroboram números próximos. Por exemplo:

\section{O Desaparecimento de Animais de Estimação no Brasil}

O desaparecimento de animais de estimação é um problema recorrente no Brasil e no mundo. Estudo conduzido pela Mars Petcare em 2022 apontou que “um em cada três animais de estimação está desabrigado no mundo” (Mars, 2022). Esse dado evidencia a magnitude da questão e reforça a necessidade de soluções que facilitem a comunicação entre tutores e comunidade. Além do sofrimento emocional e da atribulação enfrentada pelos tutores, a perda de animais pode gerar impactos sociais e de saúde pública. Animais soltos nas ruas podem se tornar vetores de zoonoses, contribuir para acidentes de trânsito e agravar problemas sanitários, sobretudo em áreas urbanas com grande densidade populacional.

No Brasil, a situação também é alarmante. Estimativas recentes mostram que “mais de 25\% dos cães e gatos estão abandonados ou perdidos” (Valor Econômico, 2024). Esses números indicam não apenas a dimensão do problema, mas também a necessidade de políticas públicas e soluções tecnológicas inovadoras. Socialmente, o abandono de \textit{pets} também reflete a fragilidade de valores relacionados à responsabilidade e ao cuidado. (UFPB, 2024). Esse cenário justifica a adoção de ferramentas digitais que otimizem a busca por animais desaparecidos e ampliem as possibilidades de reencontro entre tutores e seus animais, além de uma forma de informatização da sociedade sobre o problema que o abandono se torna neste contexto.

\section{Plataformas digitais e sistemas web}

Nos últimos anos, cresceram as iniciativas digitais para auxiliar no reencontro de animais, aproveitando o alcance das redes sociais e o avanço de tecnologias web. A plataforma PawBoost, por exemplo, afirma: “Nós ajudamos a reunir mais de 2 milhões de \textit{pets} perdidos com suas famílias” (PawBoost, 2024). Esse exemplo demonstra o impacto positivo que plataformas web podem ter quando aplicam notificações, geolocalização e mobilizam comunidades online. A eficácia dessas soluções está justamente em seu poder de rede, pois permitem que centenas de pessoas recebam informações em tempo real, aumentando a probabilidade de localização rápida.

Outra iniciativa relevante é a parceria entre a plataforma de auxílio Petco Love Lost e a Nextdoor (empresa que opera um serviço de rede social hiperlocal para bairros), que “permite que milhões de vizinhos recebam alertas imediatos quando um animal de estimação se perde na comunidade” (Petco Love Lost, 2023). Essa integração mostra como sistemas web podem usar dados de vizinhança para potencializar o alcance das buscas, criando um ecossistema colaborativo entre tutores, vizinhos e organizações de proteção animal. Como destaca Castells (2003), “a internet tornou-se a principal infraestrutura de comunicação em rede, transformando a forma como as sociedades funcionam”. No contexto do reencontro de animais, isso significa usar a conectividade digital para enfrentar um problema tradicional com novos recursos.

\section{Usabilidade e acessibilidade em sistemas web}
Um sistema web destinado ao reencontro de animais deve priorizar simplicidade e acessibilidade, dado que os tutores se encontram em situações emocionais de estresse. Nielsen (1994) estabelece que “a usabilidade refere-se a atributos de qualidade que avaliam a facilidade de uso de interfaces de usuário” (NIELSEN NORMAN GROUP, 2020). Isso significa que a interface deve ser intuitiva, de fácil aprendizado e fornecer feedback claro em todas as etapas, desde o cadastro do animal até a emissão de alertas de desaparecimento.

A acessibilidade também é obrigatória e se relaciona diretamente com o princípio da inclusão digital. A W3C afirma que “a Web é fundamentalmente projetada para funcionar para todas as pessoas, independentemente de seu hardware, software, idioma, cultura, localização ou capacidade” (W3C, 2019). Para tanto, recomenda-se que sistemas web adotem boas práticas como textos alternativos em imagens, contraste adequado entre texto e fundo, navegação por teclado e design responsivo. Segundo a norma ISO 9241-11:2018, “a usabilidade é a extensão na qual um sistema pode ser usado por usuários específicos para alcançar objetivos específicos com eficácia, eficiência e satisfação em um contexto de uso especificado” (ISO, 2018). Isso reforça a importância de realizar testes de usabilidade com o público-alvo, avaliando o desempenho real do sistema em situações práticas.

Outro ponto importante é o fator emocional. Estudos em Interação Humano-Computador destacam que sistemas projetados para contextos de estresse devem ser ainda mais claros, reduzindo frustração e aumentando a confiança do usuário. “O design emocional é essencial porque produtos que evocam confiança e empatia melhoram a experiência de uso”(Norman; Nielsen, 2013).

\section{Proteção de dados e legislação}
O tratamento de dados pessoais em sistemas web no Brasil é regido pela Lei Geral de Proteção de Dados Pessoais (LGPD). O Artigo 1º da Lei nº 13.709/2018 estabelece: “Esta Lei dispõe sobre o tratamento de dados pessoais, inclusive nos meios digitais, por pessoa natural ou por pessoa jurídica de direito público ou privado, com o objetivo de proteger os direitos fundamentais de liberdade e de privacidade” (Planalto, 2018).

Isso implica que o sistema web deve obter consentimento explícito dos usuários para coleta de dados, além de oferecer meios claros de exclusão e atualização das informações. O Artigo 18 complementa: “O titular dos dados pessoais tem direito a obter do controlador […] a correção de dados incompletos, inexatos ou desatualizados” (Planalto, 2018). Em outras palavras, qualquer sistema que lide com dados de tutores, como nome, telefone, e-mail e localização do animal, precisa garantir mecanismos de segurança e transparência.

A LGPD internaliza a orientação constitucional de que a proteção do consumidor e a dignidade da pessoa humana são erigidas como princípios da ordem econômica . As suas disposições preliminares enunciam que a disciplina da proteção de dados pessoais tem como objetivo proteger os direitos fundamentais e o livre desenvolvimento da personalidade (art. 1º), repetindo-os como um dos seus fundamentos ao lado do desenvolvimento econômico-tecnológico e da inovação (art. 2º)(BIONI; Bruno, 2019). No caso de um sistema para reencontro de animais, é essencial equilibrar o compartilhamento de informações (necessário para localizar o animal) com a proteção da privacidade dos usuários.

\section{Considerações finais}
A revisão de literatura evidencia que o uso de sistemas web dedicados ao reencontro de animais perdidos aumenta significativamente as chances de sucesso em comparação com métodos tradicionais. Exemplos internacionais como PawBoost e Petco Love Lost demonstram que o uso de geolocalização, alertas comunitários e redes de vizinhança são práticas eficazes. No Brasil, iniciativas locais podem se beneficiar dessa abordagem, desde que respeitem princípios de usabilidade, acessibilidade e a legislação vigente sobre dados pessoais.

Assim, a fundamentação teórica mostra que há espaço para a construção de uma solução inovadora: um sistema web que centralize informações sobre animais desaparecidos, integre funcionalidades de alerta, garanta acessibilidade universal e respeite integralmente a LGPD. Ao unir tecnologia e responsabilidade social, esse tipo de sistema tem potencial para reduzir o tempo de reencontro, diminuir impactos emocionais nos tutores e colaborar com a sociedade no enfrentamento do problema do desaparecimento de animais.

\chapter{Metodologia}
\label{chapter:metodologia}
\section{Delineamento da Pesquisa}

\section{Participantes}


\section{Local de Coleta de Dados}


\section{Equipamentos e Recursos}




\section{Instrumentos para coleta de dados}


\subsection{Instrumento A}
 

\subsection{Ferramentas Utilizadas}


\section{Procedimentos de Coleta de Dados}


\section{Procedimentos de Análise de Dados}





%\chapter{Resultados Esperados}
%\label{chapter:resultados-esperados}
%\input{tex/resultados-esperados}

%\chapter{Cronograma}
%\label{chapter:cronograma}
%Para execução das atividades a serem realizadas até a defesa do Trabalho de Conclusão do Curso teremos as seguintes etapas, conforme \autoref{tab:cronograma} abaixo.

\begin{landscape}
	\begin{table}[!htbp]
		\footnotesize %Habilitar se a tabela estiver muito comprida.
		\centering
		\caption{Cronograma de Atividades - Ano I e Ano II}
		\setlength{\tabcolsep}{0pt}
		\begin{tabular}{lcccccccccccccccccccc}
			
			\hline
			\multicolumn{ 1}{c}{\textbf{\cellcolor{gray!50}Atividade}} \cellcolor{gray!50}& & \multicolumn{19}{c}{\textbf{\cellcolor{gray!50}Período}}  \\ \cline{ 3 - 21}
			\cellcolor{gray!50}& \hspace{0.3cm}\cellcolor{gray!50} &
			\multicolumn{9}{c}{\textbf{\cellcolor{gray!50}Ano I}} & \multicolumn{10}{c}{\textbf{\cellcolor{gray!50}Ano II}} \\
			\cellcolor{gray!50}& \hspace{0.3cm}\cellcolor{gray!50} &
			\textbf{\cellcolor{gray!50}Fev}\cellcolor{gray!50} & \hspace{0.3cm}\cellcolor{gray!50} & \textbf{\cellcolor{gray!50}Mar} & \hspace{0.3cm}\cellcolor{gray!50} & \textbf{\cellcolor{gray!50}Abr} & \hspace{0.3cm}\cellcolor{gray!50} & \textbf{\cellcolor{gray!50}Mai} & \hspace{0.3cm}\cellcolor{gray!50} & \textbf{\cellcolor{gray!50}Jun} & \hspace{0.3cm}\cellcolor{gray!50} & \textbf{\cellcolor{gray!50}Ago} & \hspace{0.3cm}\cellcolor{gray!50} & \textbf{\cellcolor{gray!50}Set} & \hspace{0.3cm}\cellcolor{gray!50} & \textbf{\cellcolor{gray!50}Out} & \hspace{0.3cm}\cellcolor{gray!50} & \textbf{\cellcolor{gray!50}Nov} & \hspace{0.3cm}\cellcolor{gray!50} & \textbf{\cellcolor{gray!50}Dez} \\ \hline
			
			Definição do Tema, Introdução, Objetivos e Problema & \hspace{0.3cm} & $\bullet$ & \hspace{0.3cm} &  & \hspace{0.3cm} &  & \hspace{0.3cm} &   & \hspace{0.3cm} &  & \hspace{0.3cm} &  & \hspace{0.3cm} &  & \hspace{0.3cm} &  & \hspace{0.3cm} &  & \hspace{0.3cm} &  \\
			
			\rowcolor{black!10}[0pt][0pt] Elaboração da Justificativa e Hipóteses & \hspace{0.3cm} & $\bullet$ & \hspace{0.3cm} &  & \hspace{0.3cm} &  & \hspace{0.3cm} &   & \hspace{0.3cm} &  & \hspace{0.3cm} &  & \hspace{0.3cm} &  & \hspace{0.3cm} &  & \hspace{0.3cm} &  & \hspace{0.3cm} &  \\ 
			
			Revisão Bibliográfica / Teórica da pesquisa & \hspace{0.3cm}& $\bullet$  & \hspace{0.3cm} &  & \hspace{0.3cm} &  & \hspace{0.3cm} &   & \hspace{0.3cm} &  & \hspace{0.3cm} &  & \hspace{0.3cm} &  & \hspace{0.3cm} &  & \hspace{0.3cm} &  & \hspace{0.3cm} &  \\ 
			
			\rowcolor{black!10}[0pt][0pt] Definição da Metodologia& \hspace{0.3cm} &  & \hspace{0.3cm} & $\bullet$ & \hspace{0.3cm} &  & \hspace{0.3cm} &   & \hspace{0.3cm} &  & \hspace{0.3cm} &  & \hspace{0.3cm} &  & \hspace{0.3cm} &  & \hspace{0.3cm} &  & \hspace{0.3cm} & \\ 
			
			Elaboração dos Instrumentos de Coleta de Dados & \hspace{0.3cm} &  & \hspace{0.3cm} & $\bullet$ & \hspace{0.3cm} & $\bullet$ & \hspace{0.3cm} &   & \hspace{0.3cm} &  & \hspace{0.3cm} &  & \hspace{0.3cm} &  & \hspace{0.3cm} &  & \hspace{0.3cm} &  & \hspace{0.3cm} & \\
			
			\rowcolor{black!10}[0pt][0pt]Revisão do Projeto & \hspace{0.3cm} &  & \hspace{0.3cm} & $\bullet$ & \hspace{0.3cm} & $\bullet$ & \hspace{0.3cm} &  $\bullet$ & \hspace{0.3cm} &  & \hspace{0.3cm} &  & \hspace{0.3cm} &  & \hspace{0.3cm} &  & \hspace{0.3cm} &  & \hspace{0.3cm} &   \\ 
			
			Entrega do Projeto & \hspace{0.3cm} &  & \hspace{0.3cm} &  & \hspace{0.3cm} &  & \hspace{0.3cm} &   & \hspace{0.3cm} & $\bullet$ & \hspace{0.3cm} & $\bullet$ & \hspace{0.3cm} & $\bullet$ & \hspace{0.3cm} &  & \hspace{0.3cm} &  & \hspace{0.3cm} &  \\
			
			\rowcolor{black!10}[0pt][0pt]Defesa do Projeto & \hspace{0.3cm} &  & \hspace{0.3cm} &  & \hspace{0.3cm} &  & \hspace{0.3cm} & $\bullet$  & \hspace{0.3cm} &  & \hspace{0.3cm} &  & \hspace{0.3cm} &  & \hspace{0.3cm} &  & \hspace{0.3cm} &  & \hspace{0.3cm} &  \\ 
			
		\end{tabular} \\
		\fonte {Elaboração Própria (2017)}
		\label{tab:cronograma}
	\end{table}
\end{landscape}



\chapter{Resultados e Discussão}
\label{chapter:resultados-discussao}
\section{Resultados Esperados}

Por meio deste trabalho, busca-se oferecer uma solução tecnológica que facilite o reencontro entre animais perdidos e seus tutores, alcançando tanto pessoas que perderam seus \textit{pets} quanto aquelas que encontram animais nas ruas e desejam ajudar. Espera-se que o sistema se torne uma ferramenta acessível e intuitiva, capaz de ser utilizada por diferentes perfis de usuários, independentemente de sua familiaridade com tecnologia, tornando-se um recurso efetivo para a comunidade em geral.

Do ponto de vista técnico, espera-se validar a aplicação de metodologias ágeis e \textit{Design Thinking} no desenvolvimento de sistemas voltados para problemas sociais, demonstrando que soluções tecnológicas centradas no usuário podem ser desenvolvidas de forma organizada e iterativa, gerando resultados práticos e funcionais.

Por fim, almeja-se que este trabalho sirva como referência acadêmica e prática para iniciativas similares, demonstrando a viabilidade de sistemas \textit{web} dedicados ao reencontro de animais e inspirando o desenvolvimento de ferramentas que unam tecnologia e responsabilidade social em benefício da comunidade e dos animais.

\chapter{Conclusão}
\label{chapter:conclusao}
\input{tex/conclusao}




% ---
% Finaliza a parte no bookmark do PDF, para que se inicie o bookmark na raiz
% ---
\bookmarksetup{startatroot}% 
% ---

% ----------------------------------------------------------
% ELEMENTOS PÓS-TEXTUAIS
% ----------------------------------------------------------
\postextual

% ----------------------------------------------------------
% Referências bibliográficas Título da Pesquisa em Inglês
% ----------------------------------------------------------
\bibliography{references}

% ---------------------------------------------------------------------
% GLOSSÁRIO
% ---------------------------------------------------------------------

% Arquivo que contém as definições que vão aparecer no glossário
%\input{tex/glossario}
% Comando para incluir todas as definições do arquivo glossario.tex
%\glsaddall
% Impressão do glossário
%\printglossaries

% ----------------------------------------------------------
% Apêndices
% ----------------------------------------------------------

% ---
% Inicia os apêndices
% ---
\begin{apendicesenv}

% Apêndice A
\chapter{Título do Apêndice A}
\label{chapter:apendicea}

% Apêndice B
\chapter{Título do Apêndice B}
\label{chapter:apendiceb}

\end{apendicesenv}
% --- 


% ----------------------------------------------------------
% Anexos
% ----------------------------------------------------------

% ---
% Inicia os anexos
% ---
\begin{anexosenv}

\chapter{Título do Anexo A} 
\label{chapter:anexoa}

% CONFIGURAÇÕES DE PDF
\section{CONFIGURAÇÕES DE PDF}

%inserindo uma página em branco depois da página 1
%\includepdf[pages={1,{},2-10}]{arquivo03.pdf}

%inserindo múltiplas páginas numa única página
%\includepdf[pages={286-291},nup=2x3]{arquivo05.pdf}

\chapter{Título do Anexo B} 
\label{chapter:anexob}

\end{anexosenv}
% ---


\end{document}